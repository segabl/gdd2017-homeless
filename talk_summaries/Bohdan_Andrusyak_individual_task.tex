\documentclass[a4paper]{article}

%% Language and font encodings
\usepackage[english]{babel}
\usepackage[utf8x]{inputenc}
\usepackage[T1]{fontenc}

%% Sets page size and margins
\usepackage[a4paper,top=2cm,bottom=2cm,left=3cm,right=3cm,marginparwidth=1.75cm]{geometry}

%% Useful packages
\usepackage{amsmath}
\usepackage{graphicx}
\usepackage[colorinlistoftodos]{todonotes}
\usepackage[colorlinks=true, allcolors=blue]{hyperref}

\providecommand{\keywords}[1]{\textbf{\textit{Tags:}} #1}
\providecommand{\talkurl}[1]{\textbf{\textit{Url:}} #1}
\providecommand{\track}[1]{\textbf{\textit{Track:}} #1}
\providecommand{\speaker}[1]{\textbf{\textit{Speaker:}} #1}



\title{'The Armello' Postmortem: A Journey of Spirit \& Peril by Trent Kusters}
\author{Author: Bohdan Andrusyak}

\begin{document}
\maketitle

\begin{keywords} GDC 2016, Armelo, Independent Games, League of Geeks \end{keywords}

\begin{track} GDC 2016 - Independent Games Summit ) \end{track}

\begin{talkurl}  \url{http://www.gdcvault.com/play/1023416/-The-Armello-Postmortem-A} \end{talkurl}

\begin{speaker}Trent Kusters, League of Geeks \end{speaker}


\begin{abstract}
Armello is a digital role-playing strategy board game developed by Australian independent game studio League of Geeks, that was described as "Game of Thrones, Only With Animals" \cite{kotaku}. In his talk Trent Kusters, co-founder of the League of Geeks, describes the process of development from idea creation to publishing the game.  

\end{abstract}

\section{Summary of Talk}
Trent Kusters, co-founder and director of League of Geeks, describes the way that he and his team has to pass in order to develop and publish Armello. He focuses his speech on how to get financing for the project and marketing it to the community. In this talk Trent Kusters gives advice for other developers, while also acknowledging mistakes that he and his team made. 

\subsection{League of Geeks}
When Trent Kusters and his friends found League of Geeks, they had no money to start the project. Thus they developed system that would offer other people share of their profit in exchange for the work on the game. They called it "Point System", because for every accomplished task people would receive certain amount of points. This system allowed League of Geeks to cooperate with developers through out whole world. 

\subsection{The idea of Armello}
After making network of developers they established that they want to make "The three Cs" game, which means cultural, critical and commercial successful game. Trent Kusters and his team decided to develop the digital board game, because the second generation of iPad was just released and the market for digital board games was nearly empty.

\subsection{Funding}
The main source of funding for the project was Kickstarter campaign. They collected over 300.000 AUD (230.000 USD). 
In this talk Trent Kusters explains whole process of rising money on Kickstarter. He explains that Kickstarte page must offer to the people everything that they would expect from the game. It is very important, because the practice show that if the project does not collect 40\% of the goal sum, then this project is going to fail. To ensure the Kickstarter campaign will be successful League of Geeks did everything to ensure the proper press coverage. 

\subsection{Marketing}
Trent Kusters provides peice of advice for marketing campaign. First, choose  several big publication and dedicate work to those publications. Second, communicate with people face to face when showing the game.Third, colaborate with other developers. And finaly, don;t waste people time.

\subsection{Mistakes}
On their way to publish the game Trent Kusters and his team made several mistakes. They acknowledge those mistakes and shared it with everyone so other won't make them.  One of the biggest mistakes was to announce game to early. Then they spent too much money on localization. After that they did not do enough work in post launch campaign. But even with those mistakes Armelo was successful game.

\section{Overview and Relevance}
'The Armello' Postmortem: A Journey of Spirit \& Peril by Trent Kusters provide good insight on the process of development game from business perspective. And it shows that assess to crowdfunding and proper self-marketing can give every idea chance to success.

\subsection{Crowdfunding}
Armello is a good example of game that was mainly sponsored by gaming community. However, it is not a lonely example, the analyze of American market of video game industry \cite{report} stats that in 2015, 678 games sought funding from supporters with over \$160 million pledged towards games on the crowdfunding site. But there is drawback to this - more and more companies are trying their luck on crowdfunding sites thus making it very hard to get the top. Thus developers must used effective strategies and tactics such as the ones described by Thomas Bidaux in his talk Crowdfunding 301: State of Play, Best Practices & Advanced Tactics by Thomas Bidaux \cite{talk0}.
\subsection{Effective Marketing}
Successful crowdfunding requires proper marketing campaign and development team of Armelo did everything to make sure people know about Armelo. Social media platforms such as Facebook and Twitter have made it easier for companies to generate buzz around their games. Still, social media buzz is not enough for successful marketing campaign, companies need to use a lot of internal and external resources for that and it can be very hard for companies with limited finances. Chris Dwyer addresses this problem in his talk Cost Effective Marketing & PR for Indies \cite{talk1}. 
\subsection{Self Publishing and Early Access}
Appearance of digital distribution allowed small companies to compete against big studios. What is more Early Access and Steam Greenlight features that reduced barriers to entry for
independent developers. Early access allows developers to gain early feedback from potential audience.  League of Geeks utilized Steam's Early Access feature while developing Armello, they used it for finding bugs and understanding what features people like.  
\subsection{Further reading}
There is a good amount of talks that provide information about whole cycle of game development from different perspectives. Such talks include: Darkest Dungeon: A Design Post-Mortem by Tyler Sigman \cite{talk2}, Virtual Rick-ality' Postmortem: VR Lessons *burrrp* Learned by Devin Reimer, Alex Schwartz \cite{talk3} and An RTS Without Guns - OFFWORLD Trading Company by Soren Johnson 

\renewcommand{\refname}{\section{References and Further Sources}}
\begin{thebibliography}{1}


\bibitem{kotaku}
Kotaku.com
\\\texttt{https://kotaku.com/5944840/its-like-game-of-thrones-only-with-animals-i-think}
\bibitem{report}
Analyzing the American Video Game Industry 2016
\\\texttt{http://www.theesa.com/wp-content/uploads/2017/02/ESA-VG-Industry-Report-2016-FINAL-Report.pdf}
\bibitem{talk0}
Crowdfunding 301: State of Play, Best Practices & Advanced Tactics by Thomas Bidaux
\\\texttt{https://www.gdcvault.com/play/1023775/Crowdfunding-301-State-of-Play}
\bibitem{talk1}
Cost Effective Marketing & PR for Indies by Chris Dwyer
\\\texttt{https://www.gdcvault.com/play/1024116/(Opportunity)-Cost-Effective-Marketing-PR}
\bibitem{talk2}
Darkest Dungeon: A Design Post-Mortem by Tyler Sigman
\\\texttt{https://www.gdcvault.com/play/1023435/Darkest-Dungeon-A-Design}
\bibitem{talk3}
'Rick and Morty: Virtual Rick-ality' Postmortem: VR Lessons *burrrp* Learned by Devin Reimer, Alex Schwartz
\\\texttt{https://www.gdcvault.com/play/1024740/-Rick-and-Morty-Virtual}
\bibitem{talk4}
An RTS Without Guns - OFFWORLD Trading Company by Soren Johnson
\\\texttt{https://www.gdcvault.com/play/1024297/-Offworld-Trading-Company-An}
\end{thebibliography}

\end{document}