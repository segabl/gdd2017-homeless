\documentclass[a4paper]{article}

%% Language and font encodings
\usepackage[english]{babel}
\usepackage[utf8x]{inputenc}
\usepackage[T1]{fontenc}

%% Sets page size and margins
\usepackage[a4paper,top=2cm,bottom=2cm,left=3cm,right=3cm,marginparwidth=1.75cm]{geometry}

%% Useful packages
\usepackage{amsmath}
\usepackage{graphicx}
\usepackage[colorinlistoftodos]{todonotes}
\usepackage[colorlinks=true, allcolors=blue]{hyperref}

\providecommand{\keywords}[1]{\textbf{\textit{Tags:}} #1}
\providecommand{\talkurl}[1]{\textbf{\textit{Url:}} #1}
\providecommand{\track}[1]{\textbf{\textit{Track:}} #1}
\providecommand{\speaker}[1]{\textbf{\textit{Speaker:}} #1}



\title{'Deus Ex: Breach': Experimenting Within the Boundaries of a AAA Franchise by Fleur Marty}
\author{Author: Dominic Potzinger}

\begin{document}
\maketitle

\begin{keywords} GDC 2017, Fleur Marty, Eidos-Montreal, free content, Production \& Team Management, Production, Design \end{keywords}

\begin{track} GDC 2017 - Production \& Team Management ) \end{track}

\begin{talkurl}  \url{https://www.gdcvault.com/play/1024074/-Deus-Ex-Breach-Experimenting} \end{talkurl}

\begin{speaker}Fleur Marty, Eidos-Montreal \end{speaker}


\begin{abstract}
Fleur Marty talks about her experience as a Producer for Deus Ex: Breach, the developement, and design decisions.

\end{abstract}

\section{Summary of Talk}

\subsection{Background}
Deus Ex: Breach is a gamemode developed for the game Deus Ex: Mankind Divided released in 2016. It was meant as additional content separated from the main games story arch, but was later on released as a distinct Free to Play title. While critics praised the game mode for its inventive gameplay the mode was the target of a strong negative response from fans due to its microtransaction monetization model. 

\subsection{Development}
Eidos-Montreal set up a separate team, dubbed the Live-Team, tasked with creating additional game content that would extend the games playtime. This also included post launch content that would be released later on as DLC.\\
Breach was initially meant to be a challenge mode, placing the player in environments reused from the main game but with more specialized tasks. Marty mentions time or score based objectives that would be combined with player restrictions to create special challenges.
The mode had to reuse as many assets and concepts from the main game as possible, however due to the nature of the development cycle the main game and its features were not finished at the time of development. Therefore features Breach would have relied on might not have made it into the main game. Marty gives a VR Tutorial Area as an example. \\
This concept was quickly overthrown however as it was not deemed fun or engaging enough to actually draw player attention towards it, and replaced with a more engaging concept for a mode that had its own story and artstyle while being true to the Deus Ex Franchise. The concept was approved and more resources allocated to the project. As the initial goal for the Live-Team was to create content on top of an existing game most of the teams members were either artists, level designers, or narrative designers. Therefore additional programmers and designers had to be added to the team, many of which had to be able to handle as many different tasks at once because of the limited resources available. Marty describes this task as a very tough one as people in the industry are usually pushed to specialize in a specific topic. As the team's task was to create additional content rather than creating a full new game people were hesitant to join the team as it had the reputation of only being a secondary team. According to Marty this turned out to be an advantage for them as members that joined did so because they wanted to work on the project rather than striving for glory.

\subsection{Design Decisions}
Deus Ex: Breach is meant to be played rather differently than the main game. Rather than to give the player all the freedom he wants, the goal of breach was to carefully restrict players to push them out of their comfort zone. This was done to keep players from seeing breach as just an extension of the main game and rather have them try things they might not have tried in the main game. By designing levels with a specific play-style in mind, the team tried to push players towards different play-styles for higher scores while still allowing them to finish a level in any way they want. \\
The looting system is a rather strong departure from the main line Deus Ex games. Rather than being able to loot enemies for their equipment, players of breach have to acquire gear before each mission.\\
By design, Breach's artstyle heavily relies on the color purple. This is meant to highlight the mode taking place in a virtual environment as the main game mostly avoids using purple colors with the exception of one main character and VR segments. The artstyle with its unnatural shapes and bright colors, although similar across the levels of the game, is further used to convey the differences in in-game company culture and strengthen the lore of the game. 


\section{Overview and Relevance}

\subsection{Reception}
At release, critics were positive towards Breach. IGN called the game mode a nifty little extra mode and a welcome addition, highlighting the striking visual style, unique augs and abilities exclusive to the mode \cite{ign}. Players however strongly criticized the additional monetization the gamemode promoted, drawing attention to the parallels to Free to Play titles that take advantage of their users, enticing them to spend ever more money on microtransactions.\\
Breach mode was released as a standalone game in January of 2017, receiving the same criticism overall being rated with a rating of "Mostly negatively" on steam as of the time of writing \cite{steam}.

\subsection{Monetization}
Arguably Breach's downfall was its monetization strategy. Rather than seeing the game as an addition to the main game it was seen as a way to earn additional revenue with monetization strategies mostly found in the then emerging mobile and Free to Play games market. As such the game pushed its players heavily towards spending additional money to obtain upgrades and additional gear. The games internal economy was laid out accordingly with dual currencies, one that could be earned and one that could be bought, to favour players that used actual money. As such the earnable currency was worth far less and could not be used to purchase certain items, forcing the player to either replay missions and forgo premium items or spend more money.

\renewcommand{\refname}{\section{References and Further Sources}}
\begin{thebibliography}{1}

\bibitem{ign}
  Deus Ex: Mankind Divided Review,
  Vince Ingenito,
  \url{http://www.ign.com/articles/2016/08/19/deus-ex-mankind-divided-review}
  as visited on February 1, 2018

\bibitem{steam}
  Deus Ex: Breach,
  \url{http://store.steampowered.com/app/555450/Deus_Ex_Breach/}
  as visited on February 1, 2018


\end{thebibliography}

\end{document}
